\documentclass{article}
\usepackage[utf8]{inputenc}

\topmargin=-0.45in
\evensidemargin=0in
\oddsidemargin=0in
\textwidth=6.5in
\textheight=9.0in
\headsep=0.25in


\linespread{0.9} 

\usepackage{natbib}
\usepackage{graphicx}
\begin{document}



\title{Hoja de trabajo #3}
\author{Gabriel Chavarria, 20181386, chavarria181386@unis.edu.gt}
\date{16 de agosto del 2018}

\maketitle

\section{Ejercicio \#1}
Sumar s(s(s(0))) \oplus s(s(s(s(0))))\\

\centering
s(s(s(0))) \oplus  s(s(s(s(0))))\\\medskip
s(s(s(s(0))) \oplus  s(s(s(0))))\\\medskip
s(s(s(s(s(0))) \oplus  s(s(0))))\\\medskip
s(s(s(s(s(s(0))) \oplus  (s(0))))\\\medskip
s(s(s(s(s(s(s(0\oplus0))))\\\medskip
s(s(s(s(s(s(s(0)))))))\\

\raggedright\section{Ejercicio \#2}
Definir la multiplicación para numeros naturales unarios:

\[            n\otimes m := \left\{
        \begin{array}{l l}
            0 & \mbox{si } n=0 \\
            0 & \mbox{si } m=0 \\
            0 & \mbox{si } m=0,n=0 \\
            m & \mbox{si } n=1 \\
            n & \mbox{si } m=1 \\
            s(i)\oplus(s(i)\otimes j) & \mbox{si } n=s(i) \\
        \end{array}
        \right.
\]

\section{Ejercicio \#3}

\begin{itemize}

        \item{$s(s(s(0)))\otimes 0$\\\medskip
        \underline{$s(s(s(0)))\otimes 0$ = 0} , m=0 entonces por definición es 0 }
        
        
        \item{$s(s(s(0)))\otimes s(0)$\\\medskip
        $s(s(s(0)))\otimes s(0)$\\\medskip
        s(0)=1 \\\medskip 
        s(s(s(0)))\otimes s(0)= \underline{s(s(s(0)))}}

        \item{$s(s(s(0)))\otimes s(s(0))$\\\medskip
        $s(s(s(0)))\oplus(s(s(s(0)))\otimes s(0))$\\\medskip
        $s(s(s(0)))\oplus(s(s(s(0)))$\\\medskip
        $s(s(s(s(0)))\oplus s(s(0)))$\\\medskip
        $s(s(s(s(s(0)))\oplus s(0)))$\\\medskip
        $s(s(s(s(s(s(0 \oplus 0))))))$\\\medskip
        \underline{$s(s(s(s(s(s(0))))))$} }\\\medskip
        Efectivamente 3\otimes 2= 6 \bigskip
        
    
\end{itemize}



\section{Ejercicio \#4}


\textbf{1-} {$a\oplus s(s(0))=s(s(a))$}\\
Caso base: $a=0$ \medskip

{$0\oplus s(s(0))=s(s(0))$}\\\medskip
{s(s(0))=s(s(0))}\\\medskip
Caso inductivo: a = s(i)\\\medskip
{$s(i) \oplus s(s(0))=s(s(s(i)))$}\\\medskip
{$s(s(i) \oplus s(0))=s(s(s(i)))$}\\\medskip
{$s(s(s(i \oplus 0)))=s(s(s(i)))$}\\  \medskip
\underline{$s(s(s(i)))=s(s(s(i)))$}\\\medskip

\bigskip

\raggedright\textbf{2-}{$a \otimes b = b \otimes a$}\\
Caso base: $a=0$ \\\medskip
0 \otimes  b = b \otimes 0\\
0=0 \\\bigskip


Caso inductivo: a = $s(i)\\\medskip$
$s(i)$\otimes b = b \otimes $s(i)$ \\
 $s(i) \oplus (s(i) \otimes b) = s(i) \oplus (s(i) \otimes b)$\\
{$s(i) \otimes b = s(i) \otimes b$}\\
$-s(i)= (n+1) $\\
$(n+1)\otimes b =(n+1) \otimes b $\\
\underline{$b=b$}

 
\bigskip


\raggedright\textbf{3-} {$a \otimes (b \otimes c)=(a\otimes b)\otimes c$}\\
Caso base: $a=0$ \\\medskip
{$0 \otimes (b \otimes c)=(0\otimes b)\otimes c$}\\
{$0=(0)\otimes c$}\\
{$0=0$}\\
\medskip

Caso inductivo: a = s(i)=(n+1)\\\medskip
{$s(i) \otimes (b \otimes c)=(s(i)\otimes b)\otimes c$}\\ 
$(n+1) \otimes (b \otimes c) =((n+1) \otimes b) \otimes c \\
nb \otimes c+ bc = (nb \oplus b) \otimes c \\
nbc \oplus bc = nbc \oplus bc\\
nbc \ominus  nbc \oplus bc =bc\\
bc=bc\\
c=c\\

\underline{0=0}$\\



\bigskip


\raggedright\textbf{4-}{$(a\oplus b)\otimes c = (a\otimes c) \oplus (b \otimes c)$}\\
Caso base: $a=0$ \\\medskip
$(0\oplus b)\otimes c = (0\otimes c) \oplus (b \otimes c)$\\
$(b)\otimes c = (0) \oplus (b \otimes c)$\\
$b\otimes c = b \otimes c$\\\medskip




Caso inductivo: a = (n+1)\\\medskip

$(a \otimes b) \otimes (n \oplus 1) = (a \otimes (n \oplus 1)) \oplus (b \otimes (n \oplus 1))$\\

$(a \otimes (n \oplus 1) \oplus (b \otimes (n \oplus 1)) = (an \oplus a) \oplus (bn \oplus b)$\\
$(an \oplus a) \oplus (bn \oplus b) = (an \oplus a) \oplus (bn \oplus b)$\\

$(an \ominus an) \oplus (bn \ominus bn) \oplus (a \ominus a) \oplus (b \ominus b) = 0$\\
\underline{0=0}


\bigskip
\bigskip

*(Suma:
m\oplus n= s(i \oplus j)\\
si n=s(i) )











\end{document}
