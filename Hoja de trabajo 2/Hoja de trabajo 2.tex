\documentclass{article}
\usepackage[utf8]{inputenc}

\topmargin=-0.45in
\evensidemargin=0in
\oddsidemargin=0in
\textwidth=6.5in
\textheight=9.0in
\headsep=0.25in


\linespread{0.9} 

\usepackage{natbib}
\usepackage{graphicx}
\begin{document}



\title{Hoja de trabajo 2}
\author{Gabriel Chavarria, 20181386, chavarria181386@unis.edu.gt}
\date{02 de agosto del 2018}

\maketitle

\section{Ejercicio \#1}

\raggedright Demostración por inducción:\\
$$\centering\forall\ n.\ n^3\geq n^2$$\\
\raggedright Caso Base: n=0 \\
$$(0^3) \geq (0^2)$$

$$\centering 0 \geq 0 $$\\ \textit{-Por lo tanto si cumple para el caso base.}\\




 \centering Hipótesis inicial:$$ n^3\geq n^2$$\\
\raggedright Demostrar para  n= n+1 \\ 
$$\Rightarrow (n+1)(n+1)^2 \geq (n+1)^2 $$\\
$$\Rightarrow (n+1) \geq \frac{(n+1)^2}{(n+1)^2} $$\\ 
$$\Rightarrow (n+1) \geq 1 $$\\
$$\Rightarrow (n) \geq 1-1$$ \\
$$\Rightarrow n \geq 0 $$\\


\raggedright (n\in\mathbb{N}, \mathbb{N}\geq 0)\\
-Por lo cual para (n+1) si se cumple.\\

\section{Ejercicio \#2}
\raggedright Demostrar por inducción \\

$$\centering\forall\ n.\ (1+x)^n\geq nx$$\\
\raggedright Caso Base: n=0 \\
$$ (1+x)^0\geq (0)x$$\\
$$ 1\geq (0)$$ \textit{-Por lo tanto si cumple para el caso base.}\\

\centering Hipótesis inicial:$$\Rightarrow (1+x)^n\geq nx+1$$\\
\textit{($n\in \mathbb{N}$, $x\in \mathbb{Q}$ y $x+1\geq 0$)}

\raggedright Demostrar para  n= n+1 \\ 
$$\Rightarrow (1+x)^{n+1}\geq (1+x)(nx+1)\\
\centering x+1 \geq 0\\

$$\Rightarrow (1+x)^{n+1}\geq nx+1+nx^2+x$$\\


\centering $$\Rightarrow (1+x)^{n+1}\geq 1+x(1+n)+nx^2$$\\



((nx^2) \geq 0) 

$$\Rightarrow (1+x)^{n+1}\geq 1+x(1+n)$$\\ 

\raggedright -Por lo cual para (n+1) si se cumple.\\



\end{document}
