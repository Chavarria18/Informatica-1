\documentclass{article}
\usepackage[utf8]{inputenc}
% Margins
\topmargin=-0.45in
\evensidemargin=0in
\oddsidemargin=0in
\textwidth=6.5in
\textheight=9.0in
\headsep=0.25in

\linespread{1.1} % Line spacing

\usepackage{natbib}
\usepackage{graphicx}
\begin{document}


\title{Hoja de trabajo 1}
\author{Gabriel Chavarria, 20181386, chavarria181386@unis.edu.gt}
\date{25 de Julio del 2018}

\maketitle

\section{Ejercicio 2}
\begin{enumerate}
        \item{ Conjunto de nodos: \left\lbrace 1,2,3,4,5,6\right\rbrace}
        
       
        
        \item{Conjunto de vértices:         \newline\left\lbrace<1,2>,<2,6>,<6,5>,<5,1>\right\rbrace 
        \newline\left\lbrace<1,5>,<5,6>,<6,2>,<2,1>\right\rbrace 
        \newline\left\lbrace<1,3>,<3,6>,<6,4>,<4,1>\right\rbrace 
        \newline\left\lbrace<1,4>,<4,6>,<6,3>,<3,1>\right\rbrace 
        \newline\left\lbrace <4,2>,<2,3>,<3,5>,<5,4>\right\rbrace 
        \newline \left\lbrace<3,2>,<2,4>,<4,5>,<5,3> \right\rbrace  
        
        }
        
         \end{enumerate}        
                


\section{Ejercicio 3}
\begin{enumerate}
        \item{ Una estructura de camino.}
        \item{Modelo algebraico que siga una estructura de camino, donde este permita que un mismo numero pueda salir más de  una vez y también que logre manejar el azar que se requiere.  }
        \item{Nos aseguramos que funcione haciendo que no se repita el proceso, se evita que se genere un ciclo. }
                
\end{enumerate}

\end{document}
